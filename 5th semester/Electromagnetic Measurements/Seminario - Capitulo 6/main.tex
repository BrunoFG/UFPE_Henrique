%----------------------------------------------------------------------------------------
%	PACKAGES AND THEMES
%----------------------------------------------------------------------------------------
\documentclass[aspectratio=169,xcolor=dvipsnames]{beamer}
\usetheme{SimplePlusAIC}

\usepackage[brazil]{babel}
\usepackage{hyperref}
\usepackage{graphicx} % Allows including images
\usepackage{booktabs} % Allows the use of \toprule, \midrule and  \bottomrule in tables
\usepackage{svg} %allows using svg figures
\usepackage{tikz}
\usepackage{makecell}
\newcommand*{\defeq}{\stackrel{\text{def}}{=}}

%Select the Epilogue font (requires luaLatex or XeLaTex compilers)
\usepackage{fontspec}
\setsansfont{Epilogue}[
    Path=./epilogueFont/,
    Scale=0.9,
    Extension = .ttf,
    UprightFont=*-Regular,
    BoldFont=*-Bold,
    ItalicFont=*-Italic,
    BoldItalicFont=*-BoldItalic
    ]

%----------------------------------------------------------------------------------------
%	TITLE PAGE
%----------------------------------------------------------------------------------------

\title[short title]{Resultados de Medições Diretas} % The short title appears at the bottom of every slide, the full title is only on the title page
\subtitle{Capítulo 6}

\author[Surname]{Gabriel Soares // Henrique da Silva}
\institute[UFPE]{Centro de Tecnologia e Geociências \newline Curso de Engenharia Eletrônica\newline Universidade Federal de Pernambuco}
% Your institution as it will appear on the bottom of every slide, maybe shorthand to save space


\date{23/02/2023} % Date, can be changed to a custom date
%----------------------------------------------------------------------------------------
%	PRESENTATION SLIDES
%----------------------------------------------------------------------------------------

\begin{document}

\begin{frame}[plain]
    % Print the title page as the first slide
    \titlepage
\end{frame}

\begin{frame}{Sumário}
    % Throughout your presentation, if you choose to use \section{} and \subsection{} commands, these will automatically be printed on this slide as an overview of your presentation
    \tableofcontents
\end{frame}

%------------------------------------------------
\section{Medições diretas e indiretas}
%------------------------------------------------

\begin{frame}{Medições diretas e indiretas}
    \begin{itemize}
   	 \item Medição direta
   	 \item Medição indireta

    \end{itemize}
\end{frame}

%------------------------------------------------
\section{Processo de medição}
%------------------------------------------------

\begin{frame}{Processo de medição}
    \begin{itemize}
   	 \item Condições ambientais
   	 \item Definição do mensurando
   	 \item Operador
   	 \item Procedimento de medição
   	 \item Sistema de medições
   	 \item Fontes de incerteza
    \end{itemize}
\end{frame}


%------------------------------------------------
\section{Variabilidade do mensurando}
%------------------------------------------------

\begin{frame}{Variabilidade do mensurando}
    \begin{itemize}
   	 \item Mensurando invariável
   	 \item Mensurando variável
    \end{itemize}
\end{frame}

%------------------------------------------------
\section{Incerteza de mensurando invariável}
%------------------------------------------------

\begin{frame}{Incerteza de mensurando invariável}
    \begin{itemize}
   	 \item Particularidades do processo de medição
   	 \item Dificuldade de quantificar separadamente fontes de incerteza
   	 \item Fonte de incerteza dominante
   	 \item Corrigindo erros sistemáticos
    \end{itemize}
\end{frame}

%-------------------------
\section{Grafia correta dos resultados da medição}
%------------------------------------------------

\begin{frame}{Grafia correta dos resultados da medição}
    \begin{itemize}
   	 \item Algarismos significativos
   	 \item Regras de arredondamento
   	 \item Arredondamentos nos cálculos
    \end{itemize}
\end{frame}

%------------------------------------------------
\section{Incerteza de mensurando variável}
%------------------------------------------------

\begin{frame}{Incerteza de mensurando variável}
    \begin{itemize}
   	 \item Altura de um muro variável
   	 \item Corrigindo erros sistemáticos

\\
	\begin{equation}
   	RM = \left(I + C)\right \pm \left(t * u\right)
	\end{equation}
    \end{itemize}
\end{frame}

%------------------------------------------------
\section{Fontes de incerteza múltiplas}
%------------------------------------------------

\begin{frame}{Fontes de incerteza múltiplas}
    \begin{itemize}
   	 \item Identificação das fontes  de incerteza
    	\item Interdependência e correção combinada
    	\item Procedimentos não estatísticos - "Tipo B"
    	\item Graus de liberdade efetivos

    	\begin{equation}
    	\begin{aligned}
    	&u^2_c = u^2_1 + ... + u^2_n \\
      	&\frac{u^4_c }{v_{ef}} = \frac{u^4_1 }{v_{1}} + ... + \frac{u^4_n }{v_{n}}
    	\end{aligned}   	 
	\end{equation}
	\item Incerteza expandida

	\begin{equation}
    	U = t * u_c
	\end{equation}
   	 
    \end{itemize}
\end{frame}

%------------------------------------------------
\section{Procedimento passo a passo}
%------------------------------------------------

\begin{frame}{Procedimento passo a passo}
    \begin{itemize}
   	 \item Análise do processo de medição
    	\item Identificação das fontes de incerteza
    	\item Quantificação dos efeitos sistemáticos
    	\item Quantificação dos efeitos aleatórios
    	\item Cálculo da correção combinada
    	\item Cálculo da incerteza combinada e número de graus de liberdade efetivos
    	\item Cálculo da incerteza expandida
    	\item Expressão do resultado da medição
   	 
    \end{itemize}
\end{frame}

\end{document}


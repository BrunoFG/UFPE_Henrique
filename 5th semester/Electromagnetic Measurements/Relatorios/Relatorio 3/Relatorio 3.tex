\documentclass[12pt,twoside, a4paper, twocolumn]{article}
\usepackage[utf8]{inputenc}
\usepackage[brazil]{babel}
\usepackage[margin = 0.5in]{geometry}
\usepackage{amsmath}
\usepackage{amsthm}
\usepackage{amssymb}
\usepackage{amsthm}
\usepackage{setspace}
\usepackage[americanvoltages,fulldiodes,siunitx]{circuitikz}
\usepackage{lipsum}
\usepackage{pgfplots}
\usepackage{ifthen}
\usepackage{adjustbox}
\usepackage[section]{placeins}
\usepackage{hyperref}
\usepackage{graphicx}
\usepackage{amsmath}
\usepackage{amsthm}
\usepackage{amssymb}
\usepackage{amsthm}
\usepackage{setspace}
\usepackage[americanvoltages,fulldiodes,siunitx]{circuitikz}
\usepackage{lipsum}
\usepackage{pgfplots}
\usepackage{ifthen}
\usepackage{adjustbox}
\usepackage[section]{placeins}
\usepackage{hyperref}
\usepackage{graphicx}
\usepackage{adjustbox}
 
\pgfplotsset{compat=newest}
 
\graphicspath{ {./images/} }
 
%  #1 color - optional #2 x_0 #3 y_0 #4 x_f #5 y_f #6 name - optional  #7 true if adding lines to axis
 
\newcommand{\drawvector} [9] [color=cyan] {
   \draw[line width=1.5pt,#1,-stealth](axis cs: #2, #3)--(axis cs: #4, #5) node[anchor=south west]{$#6$};
 
  
 
\ifthenelse{\equal{#7}{true}}{
   \draw[line width=1pt,#1, dashed](axis cs: #4, #5)--(axis cs: #4, 0) node[anchor= north west]{$#8$};
   \draw[line width=1pt,#1, dashed](axis cs: #4, #5)--(axis cs: 0, #5) node[anchor=south east]{$#9$};
   }
   {}
}
 
\newcommand\deriv[2]{\frac{\mathrm d #1}{\mathrm d #2}}
 
 
\title{Terceiro Relatório de Medidas Eletromagneticas}
\author{Gabriel Soares \\ Henrique da Silva}
\date{\today}
\pgfplotsset{width = 10cm, compat = 1.9}
 
 
\begin{document}
\maketitle
\pagenumbering{gobble}
\newpage
%pagenumbering{roman}
\tableofcontents
\newpage



\section{Introdução}


\subparagraph*{Neste relatório, vamos medir a capacitancia de um capacitor utilizando um filtro $RC$.}

%\subparagraph*{Todos arquivos utilizados para criar este relatorio, e o relatorio em si estão em:  \url{https://github.com/Shapis/ufpe_ee/tree/main/5th semester/lab circuitos}}




\subsection{Análise preliminar}

\subparagraph*{Construiremos um circuito $RC$ e mediremos a tensao com um osciloscópio em paralelo com o capacitor.}

\subparagraph*{E utilizaremos da seguinte relacao para medir a capacitancia:}

\begin{equation}
    \begin{aligned}
        \tau = RC \\
        C = \frac{\tau}{R}
    \end{aligned}
\end{equation}

\subparagraph*{Logo utilizaremos uma fonte geradora de onda quadrada com periodo de aproximadamente 4 $\tau$ para podermos observar claramente o padrao de carregamento e descarregamento do capacitor}

\subparagraph*{Entao mediremos o tempo necessario para que a tensao atinja $63.2\%$ do seu valor de pico para obtermos o $\tau$.}

\section{Resultados esperados}

\subparagraph*{Esperamos que os valores de capacitancia que obteremos seja coerente com o valor real e que a maior fonte de imprecisao vira da nossa medicao por cursores no osciloscopio.}

\section{Medições no laboratório}

\subparagraph*{Vamos utilizar o osciloscópio para gerar uma onda quadrada que passara por um circuito \emph{RC}. E mediremos a tensao no capacitor para fazermos a analise de tempo de subida e descida.}

\subparagraph*{Faremos isto tres vezes para tres valores de $R$ previamente conhecidos, respectivamente $14800 \varOmega$, $8200 \varOmega$ e $15 \varOmega$.}

\subparagraph*{Com estes em maos determinaremos a capacitancia do nosso capacitor.}

\subparagraph*{Apos isso, mediremos a capacitancia diretamente com um multimetro para podermos fazer a analise das discrepancias entre as duas medidas.}

\subsection{Tabela de medições}

\subsubsection{Medicoes utilizando circuito RC}

\begin{center}
    \begin{tabular}{ |c|c|c| }
        \hline
        $R \varOmega$ & $\tau (s)$ & $C nF$  \\
        $15$          & 0.0000045  & $300.0$ \\
        $8200$        & 0.00054    & $65.8$  \\
        $14800$       & 0.00076    & $51.3$  \\
        \hline
    \end{tabular}
\end{center}


\subsubsection{Medicoes utilizando multimetro}

\begin{center}
    \begin{tabular}{ |cc| }
        \hline
        $ C (nF)$         \\
        $62.37$ & $62.42$ \\
        $62.16$ & $62.27$ \\
        $62.8$  & $63.1$  \\
        $62.99$ & $62.95$ \\
        $63.38$ & $62.97$ \\
        $63.3$  & $63.4$  \\
        $63.45$ & $63.61$ \\
        $64.24$ & $63.82$ \\
        $63.32$ & $63.26$ \\
        $63.24$ & $63.09$ \\
        \hline
    \end{tabular}
\end{center}

\begin{center}
    \begin{tabular}{ |cc| }
        \hline
        Media         & $63.107$ \\
        Desvio padrão & 0.5104   \\
        \hline
    \end{tabular}
\end{center}

\section{Circuito RL}

\subparagraph*{Para um hipotetico circuito $RL$, teriamos:}

\begin{equation}
    \begin{aligned}
        \tau = \frac{L}{R} \\
        L = \tau R
    \end{aligned}
\end{equation}

\subparagraph*{Que tambem nos permitiria determinar a indutancia, a diferenca seria que neste caso multiplicariamos o $\tau$ encontrado experimentalmente por $R$ para obtermos a indutancia.}

\section{Conclusoes}

\subparagraph*{Conseguimos determinar a capacitancia com mais precisao com um resistor intermediario.}

\subparagraph*{Isto ocorre devido a maior facilidade de observacao das curvas de subida e descida da tensao no capacitor vistos no osciloscopio.}



\end{document}

\documentclass[12pt,twoside, a4paper, twocolumn]{article}
\usepackage[utf8]{inputenc}
\usepackage[brazil]{babel}
\usepackage[margin = 0.5in]{geometry}
\usepackage{amsmath}
\usepackage{amsthm}
\usepackage{amssymb}
\usepackage{amsthm}
\usepackage{setspace}
\usepackage[americanvoltages,fulldiodes,siunitx]{circuitikz}
\usepackage{lipsum}
\usepackage{pgfplots}
\usepackage{ifthen}
\usepackage{adjustbox}
\usepackage[section]{placeins}
\usepackage{hyperref}
\usepackage{graphicx}
\usepackage{amsmath}
\usepackage{amsthm}
\usepackage{amssymb}
\usepackage{amsthm}
\usepackage{setspace}
\usepackage[americanvoltages,fulldiodes,siunitx]{circuitikz}
\usepackage{lipsum}
\usepackage{pgfplots}
\usepackage{ifthen}
\usepackage{adjustbox}
\usepackage[section]{placeins}
\usepackage{hyperref}
\usepackage{graphicx}
\usepackage{adjustbox}
 
\pgfplotsset{compat=newest}
 
\graphicspath{ {./images/} }
 
%  #1 color - optional #2 x_0 #3 y_0 #4 x_f #5 y_f #6 name - optional  #7 true if adding lines to axis
 
\newcommand{\drawvector} [9] [color=cyan] {
   \draw[line width=1.5pt,#1,-stealth](axis cs: #2, #3)--(axis cs: #4, #5) node[anchor=south west]{$#6$};
 
  
 
\ifthenelse{\equal{#7}{true}}{
   \draw[line width=1pt,#1, dashed](axis cs: #4, #5)--(axis cs: #4, 0) node[anchor= north west]{$#8$};
   \draw[line width=1pt,#1, dashed](axis cs: #4, #5)--(axis cs: 0, #5) node[anchor=south east]{$#9$};
   }
   {}
}
 
\newcommand\deriv[2]{\frac{\mathrm d #1}{\mathrm d #2}}
 
 
\title{Terceiro Relatório de Medidas Eletromagnéticas}
\author{Gabriel Soares \\ Henrique da Silva}
\date{\today}
\pgfplotsset{width = 10cm, compat = 1.9}
 
 
\begin{document}
\maketitle
\pagenumbering{gobble}
\newpage
%pagenumbering{roman}
\tableofcontents
\newpage



\section{Introdução}


\subparagraph*{Neste relatório, vamos medir a capacitância de um capacitor em um filtro $RC$.}

\subparagraph*{Todos arquivos utilizados para criar este relatorio, e o relatorio em si estão em:  \url{https://github.com/Shapis/ufpe_ee/tree/main/5th semester/Electromagnetic Measurements/Relatorios}}




\subsection{Análise preliminar}

\subparagraph*{Construíremos um circuito $RC$ e mediremos a tensão com um osciloscópio com as pontas de prova sobre o capacitor.}

\subparagraph*{Utilizaremos da seguinte relação para medir a capacitância:}

\begin{equation}
    \begin{aligned}
        \tau = RC \\
        C = \frac{\tau}{R}
    \end{aligned}
\end{equation}

\subparagraph*{Logo utilizaremos uma fonte geradora de onda quadrada com período de aproximadamente 4$\tau$ para podermos observar claramente o padrão de carregamento e descarregamento do capacitor.}

\subparagraph*{Entao mediremos o tempo necessário para que a tensão atinja $63.2\%$ do seu valor de pico para obtermos o $\tau$.}

\section{Resultados esperados}

\subparagraph*{Esperamos que os valores de capacitância que obteremos sejam coerentes com o valor real e que a maior fonte de imprecisão virá pelos cursores no osciloscópio.}

\section{Medições no laboratório}

\subparagraph*{Vamos utilizar o osciloscópio para gerar uma onda quadrada que passará por um circuito $RC$. Mediremos a tensão no capacitor para fazermos a análise de tempo de subida e descida.}

\subparagraph*{Faremos isso três vezes para três valores de $R$ previamente conhecidos, respectivamente $14800 \varOmega$, $8200 \varOmega$ e $15 \varOmega$.}

\subparagraph*{Com estes em mãos, determinaremos a capacitância do nosso capacitor.}

\subparagraph*{Após isso, mediremos a capacitância diretamente com um multimetro para podermos fazer a analise das discrepâncias entre as duas medidas.}

\subsection{Tabela de medições}

\subsubsection{Medições utilizando circuito RC}

\begin{center}
    \begin{tabular}{ |c|c|c| }
        \hline
        $R (\varOmega)$ & $\tau (s)$ & $C (nF)$ \\
        $15$            & 0.0000045  & $300.0$  \\
        $8200$          & 0.00054    & $65.8$   \\
        $14800$         & 0.00076    & $51.3$   \\
        \hline
    \end{tabular}
\end{center}


\subsubsection{Medições utilizando multímetro}

\begin{center}
    \begin{tabular}{ |cc| }
        \hline
        $ C (nF)$ &         \\
        $62.37$   & $62.42$ \\
        $62.16$   & $62.27$ \\
        $62.8$    & $63.1$  \\
        $62.99$   & $62.95$ \\
        $63.38$   & $62.97$ \\
        $63.3$    & $63.4$  \\
        $63.45$   & $63.61$ \\
        $64.24$   & $63.82$ \\
        $63.32$   & $63.26$ \\
        $63.24$   & $63.09$ \\
        \hline
    \end{tabular}
\end{center}

\begin{center}
    \begin{tabular}{ |cc| }
        \hline
        Média         & $63.107 nF$ \\
        Desvio padrão & $0.5104 nF$ \\
        \hline
    \end{tabular}
\end{center}

\section{Circuito RL}

\subparagraph*{Para um hipotético circuito $RL$, teríamos:}

\begin{equation}
    \begin{aligned}
        \tau = \frac{L}{R} \\
        L = \tau R
    \end{aligned}
\end{equation}

\subparagraph*{Que também nos permitiria determinar a indutância, a diferença seria que nesse caso multiplicaríamos o $\tau$ encontrado experimentalmente por $R$ para obtermos a indutância.}

\section{Conclusões}

\subparagraph*{Conseguimos determinar a capacitância com mais precisão com um resistor intermediário. Isso ocorre devido à maior facilidade de observação das curvas de subida e descida da tensão no capacitor vistos no osciloscópio.}

\subparagraph*{Mediu-se, portanto, uma capacitância usando um engenhoso método. É por meio de engenhosos pensamentos como esses que surgem e desenvolvem-se os sistemas de medição.}



\end{document}

\documentclass[12pt,twoside, a4paper, twocolumn]{article}
\usepackage[utf8]{inputenc}
\usepackage[brazil]{babel}
\usepackage[margin = 0.5in]{geometry}
\usepackage{amsmath}
\usepackage{amsthm}
\usepackage{amssymb}
\usepackage{amsthm}
\usepackage{setspace}
\usepackage[americanvoltages,fulldiodes,siunitx]{circuitikz}
\usepackage{lipsum}
\usepackage{pgfplots}
\usepackage{ifthen}
\usepackage{adjustbox}
\usepackage[section]{placeins}
\usepackage{hyperref}
\usepackage{graphicx}
\usepackage{amsmath}
\usepackage{amsthm}
\usepackage{amssymb}
\usepackage{amsthm}
\usepackage{setspace}
\usepackage[americanvoltages,fulldiodes,siunitx]{circuitikz}
\usepackage{lipsum}
\usepackage{pgfplots}
\usepackage{ifthen}
\usepackage{adjustbox}
\usepackage[section]{placeins}
\usepackage{hyperref}
\usepackage{graphicx}
\usepackage{adjustbox}
 
\pgfplotsset{compat=newest}
 
\graphicspath{ {./images/} }
 
%  #1 color - optional #2 x_0 #3 y_0 #4 x_f #5 y_f #6 name - optional  #7 true if adding lines to axis
 
\newcommand{\drawvector} [9] [color=cyan] {
   \draw[line width=1.5pt,#1,-stealth](axis cs: #2, #3)--(axis cs: #4, #5) node[anchor=south west]{$#6$};
 
  
 
\ifthenelse{\equal{#7}{true}}{
   \draw[line width=1pt,#1, dashed](axis cs: #4, #5)--(axis cs: #4, 0) node[anchor= north west]{$#8$};
   \draw[line width=1pt,#1, dashed](axis cs: #4, #5)--(axis cs: 0, #5) node[anchor=south east]{$#9$};
   }
   {}
}
 
\newcommand\deriv[2]{\frac{\mathrm d #1}{\mathrm d #2}}
 
 
\title{Segundo Relatório de Medidas Eletromagneticas}
\author{Gabriel Soares \\ Henrique da Silva}
\date{\today}
\pgfplotsset{width = 10cm, compat = 1.9}
 
 
\begin{document}
\maketitle
\pagenumbering{gobble}
\newpage
%pagenumbering{roman}
\tableofcontents
\newpage



\section{Introdução}


\subparagraph*{Neste relatório, vamos medir os valores de resistencia $\varOmega$ e capacitancia $F$ de resistores e capacitores, e calcularemos alguns de seus parametros estatisticos.}

\subparagraph*{Todos arquivos utilizados para criar este relatorio, e o relatorio em si estão em:  \url{https://github.com/Shapis/ufpe_ee/tree/main/5th semester/lab circuitos}}




\subsection{Analise preliminar}
\subparagraph*{}


\subparagraph*{Utilizaremos um multimetro para medir as propriedades de alguns componentes.}

\subparagraph*{Faremos $20$ medicoes em cada componente, e calcularemos a media, desvio padrao, tendencia, e correcao de cada um deles.}

\subparagraph*{Apos isto discutiremos os nossos achados.}

\section{Resultados esperados}

\subsection{Resistor}

\subparagraph*{Esperamos resultados consistentes entre as medidas, porem, tambem esperamos que a resistencia seja diferente da resistencia de fabrica.}

\subparagraph*{Isto ocorrera por desgaste dos componentes devido a seu uso de laboratorio, e tambem pela qualidade dos componentes.}

\subparagraph*{Muito provavelmente estamos fora dos padroes de confiabilidades de fabrica. Mas precisariamos ver o datasheet dos resistores em especifico para confirmar isto.}

\subsection{Capacitor}

\subparagraph*{Tudo que falamos a cima se aplica aos capacitores, mas com dois diferenciais.}

\subparagraph*{O primeiro eh que estes sao mais sensiveis ao uso, logo esperaremos discrepancias maiores entre os valores de fabrica e os de fato.}

\subparagraph*{E tambem que durante as medidas, os carregaremos e os descarregaremos, que implicara tambem em um erro sistematico adicional.}


\section{Medicoes no Laboratorio}

\subparagraph*{Utilizando um multimetro, mediremos resistencias de resistores, e capacitancias de capacitores.}

\subparagraph*{Para reduzir erros sistematicos, os encaixaremos todos componentes em um protoboard.}

\subparagraph*{E antes de fazer as medidas dos capacitores, vamos criar um circuito com um capacitor e um resistor em serie para descarregalos. Apos alguns segundos com este circuito formado, desconectaremos o circuito e faremos a medicao de fato.}

\subsection{Tabelas de medicoes}

\subsubsection{Resistores}

\subparagraph*{Mediremos tres resistores, com valores de fabrica respectivamente de: $R_1 = 10k \varOmega$, $R_2 = 22k \varOmega$, $R_3 = 15k \varOmega$.}
\begin{center}
    \begin{tabular}{ |c|c|c| }
        \hline
        $R_1$ $10k\varOmega$ & $R_2$ $22k\varOmega$ & $R_3$ $15k\varOmega$ \\
        10037 $\varOmega$    & 21932 $\varOmega$    & 14848 $\varOmega$    \\
        10037 $\varOmega$    & 21932 $\varOmega$    & 14849 $\varOmega$    \\
        10038 $\varOmega$    & 21932 $\varOmega$    & 14850 $\varOmega$    \\
        10038 $\varOmega$    & 21932 $\varOmega$    & 14849 $\varOmega$    \\
        10038 $\varOmega$    & 21932 $\varOmega$    & 14850 $\varOmega$    \\
        10037 $\varOmega$    & 21933 $\varOmega$    & 14849 $\varOmega$    \\
        10037 $\varOmega$    & 21933 $\varOmega$    & 14849 $\varOmega$    \\
        10037 $\varOmega$    & 21931 $\varOmega$    & 14850 $\varOmega$    \\
        10037 $\varOmega$    & 21931 $\varOmega$    & 14850 $\varOmega$    \\
        10037 $\varOmega$    & 21930 $\varOmega$    & 14848 $\varOmega$    \\
        10036 $\varOmega$    & 21932 $\varOmega$    & 14849 $\varOmega$    \\
        10037 $\varOmega$    & 21932 $\varOmega$    & 14849 $\varOmega$    \\
        10037 $\varOmega$    & 21932 $\varOmega$    & 14849 $\varOmega$    \\
        10037 $\varOmega$    & 21932 $\varOmega$    & 14849 $\varOmega$    \\
        10038 $\varOmega$    & 21934 $\varOmega$    & 14849 $\varOmega$    \\
        10036 $\varOmega$    & 21934 $\varOmega$    & 14850 $\varOmega$    \\
        10036 $\varOmega$    & 21934 $\varOmega$    & 14849 $\varOmega$    \\
        10037 $\varOmega$    & 21933 $\varOmega$    & 14849 $\varOmega$    \\
        10036 $\varOmega$    & 21934 $\varOmega$    & 14849 $\varOmega$    \\
        10036 $\varOmega$    & 21932 $\varOmega$    & 14848 $\varOmega$    \\

        \hline
    \end{tabular}
\end{center}

\begin{center}
    \begin{tabular}{ |c|c|c|c| }
        \hline
                      & $R_1$ $10k\varOmega$ & $R_2$ $22k\varOmega$ & $R_3$ $15k\varOmega$ \\
        Media         & 10037                & 21932                & 14849                \\
        Desvio Padrao & 0.68633              & 1.0954               & 0.64072              \\
        Tendencia     & 36.950               & -67.600              & -150.90              \\
        Correcao      & -36.950              & 67.600               & 150.90               \\


        \hline
    \end{tabular}
\end{center}

\subsubsection{Capacitores}

\subparagraph*{Mediremos tres capacitores, com valores de fabrica respectivamente de: $C_1 = 100 nF$, $C_2 = 47 nF$, $R_3 = 10 nF$.}
\begin{center}
    \begin{tabular}{ |c|c|c| }
        \hline
        $C_1 = 100 nF$ & $C_2 = 47 nF$ & $R_3 = 10 nF$ \\
        46.31 $nF$     & 55.92 $nF$    & 12.74 $nF$    \\
        46.45 $nF$     & 55.70 $nF$    & 12.72 $nF$    \\
        46.34 $nF$     & 55.66 $nF$    & 12.77 $nF$    \\
        46.34 $nF$     & 55.87 $nF$    & 12.76 $nF$    \\
        46.25 $nF$     & 56.09 $nF$    & 12.78 $nF$    \\
        46.36 $nF$     & 55.85 $nF$    & 12.77 $nF$    \\
        46.21 $nF$     & 55.90 $nF$    & 12.74 $nF$    \\
        46.32 $nF$     & 55.76 $nF$    & 12.80 $nF$    \\
        46.30 $nF$     & 55.94 $nF$    & 12.83 $nF$    \\
        46.54 $nF$     & 55.72 $nF$    & 12.84 $nF$    \\
        46.54 $nF$     & 55.69 $nF$    & 12.79 $nF$    \\
        47.01 $nF$     & 55.78 $nF$    & 12.81 $nF$    \\
        46.70 $nF$     & 55.75 $nF$    & 12.78 $nF$    \\
        46.82 $nF$     & 55.85 $nF$    & 12.80 $nF$    \\
        46.75 $nF$     & 55.82 $nF$    & 12.81 $nF$    \\
        46.64 $nF$     & 55.43 $nF$    & 12.79 $nF$    \\
        46.71 $nF$     & 55.40 $nF$    & 12.76 $nF$    \\
        46.76 $nF$     & 55.39 $nF$    & 12.73 $nF$    \\
        46.85 $nF$     & 55.64 $nF$    & 12.69 $nF$    \\
        46.81 $nF$     & 55.68 $nF$    & 12.68 $nF$    \\

        \hline
    \end{tabular}
\end{center}

\begin{center}
    \begin{tabular}{ |c|c|c|c| }
        \hline
                      & $C_1 100 nF$ & $C_2 47 nF$ & $C_3 10 nF$ \\
        Media         & 46.55        & 55.74       & 12.77       \\
        Desvio Padrao & 0.2401       & 0.1819      & 0.0430      \\
        Tendencia     & -53.45       & 8.742       & 2.770       \\
        Correcao      & 53.45        & -8.742      & -2.770      \\
        \hline
    \end{tabular}
\end{center}

\section{Conclusoes}

\subparagraph*{Obtivemos desvios padroes baixos para nossos componentes. Porem espeficiamente no caso dos capacitores as tendencias foram bastante elevadas.}

\subparagraph*{O que indica que uma calibracao seja necessaria.}


\end{document}
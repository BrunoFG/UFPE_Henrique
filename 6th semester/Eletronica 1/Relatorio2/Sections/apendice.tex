\section{Apêndice}

Abaixo se encontra o código utilizado para a análise simbólica e numérica do circuito.

\begin{python}
    import matplotlib.pyplot as plt
    import sympy as smp
    from sympy import *

    # Definindo as variaveis simbolicas
    Vo, Vi, Va, Vc, R1, R2, Rp, Cp, A, w, j, Hjw, wp, wc, K = smp.symbols(
    'V_o V_i V_a V_c R_1 R_2 R_p C_p A w j H_jw w_p, w_c, K', real=True)

    # Analise nodal do circuito

    eq1 = smp.Eq((Va - Vi)/R1 + (Va - Vo)/R2 + Va/(Rp + (1/(j * w * Cp))), 0)
    eq2 = smp.Eq(Va * ((1/(j * w * Cp)))/(((1/(j * w * Cp))) + Rp), -Vc)
    eq3 = smp.Eq(A * Vc, Vo)
    # print('Equacoes em latex:')
    # smp.pprint(smp.latex(eq1))
    # smp.pprint(smp.latex(eq2))
    # smp.pprint(smp.latex(eq3))
    # print("")

    print("Equacoes do circuito:")
    print("Equacao 1:")
    smp.pprint(eq1)
    print("Equacao 2:")
    smp.pprint(eq2)
    print("Equacao 3:")
    smp.pprint(eq3)
    print("")

    sols = smp.solve([eq1, eq2, eq3], [Va, Vc, Vo])

    print("Solucao para Vo:")
    smp.pprint(sols[Vo])
    # print("")
    # smp.pprint(smp.latex(sols[Vo]))
    print("")

    print("Aqui fazemos a seguinte simplificacao:")
    print("Rp >> R1 , Rp >> R2, e A >> 1")

    Vo = (-A * R2 * Vi)/((Rp*(R1 + R2))*j*w*Cp + A*R1)

    print("Equacao de Vo simplificada:")
    smp.pprint(Vo)
    # smp.pprint(smp.latex(Vo))
    print("")

    eqHjw = smp.Eq(Hjw, Vo/Vi)

    print("Equacao de Hjw:")
    smp.pprint(eqHjw)
    print("")

    print("Resolvendo a equacao Hjw e colocando no formato canonico da um fitro passa baixa (-K wp / (jw + wp)) obtemos o seguinte:")

    eqHjw = smp.Eq(Hjw, -(K * wc) / (I * w + wc))

    Hjw = -(K * wc) / (I * w + wc)

    smp.pprint(eqHjw)

    print("")

    eqwp = smp.Eq(wp, 1/(Rp*Cp))
    smp.pprint(eqwp)

    eqK = smp.Eq(K, R2/R1)
    smp.pprint(eqK)

    eqwc = smp.Eq(wc, (A*wp)/(1 + K))
    smp.pprint(eqwc)

    print("")

    # Hjw = Hjw.subs({K: R2/R1, wc: A*wp/(1+K)})

    print("Valor absoluto de Hjw:")
    Hjw_abs = smp.Abs(Hjw)
    smp.pprint(Hjw_abs)
    # smp.pprint(smp.latex(Hjw_abs))
    print("")


    print("Exemplo 1:")
    print("Para R1 = 4.7E3 ohms, R2 = 2.2E4 ohms, wp = 2E1 pi e A = 1E5\n")

    eqK1 = eqK.subs({R1: 4.7E3, R2: 2.2E4})
    K1 = smp.solve(eqK1, K)[0]
    smp.pprint(eqK1)
    print("")

    eqwc1 = eqwc.subs({A: 1E5, wp: 20 * smp.pi, K: K1}).evalf()
    wc1 = smp.solve(eqwc1, wc)[0]
    smp.pprint(eqwc1)
    print("")

    print("Valores absolutos para w = [0.5, 1, 2, 4, 10, 20, 40]*wc:")

    # ex1interval = [0.02, 0.01, 0.05, 0.2, 0.5, 1, 2, 4, 10, 20, 40]
    ex1interval = [0.5, 1, 2, 4, 10, 20, 40]

    ex1Vals = []

    for val in ex1interval:
    temp = Hjw_abs.subs({w: wc1 * val, wc: wc1, K: K1})
    ex1Vals.append(temp)
    print('para freq:', round(((val*wc1)/(2*smp.pi)).evalf(), 2),
    '\ttemos:', round(temp, 2))

    print("")

    print("Exemplo 2:")
    print("Para R1 = 4.7E3 ohms, R2 = 5.6E5 ohms, wp = 2E1 pi e A = 1E5\n")

    eqK2 = eqK.subs({R1: 4.7E3, R2: 5.6E5})
    K2 = smp.solve(eqK2, K)[0]
    smp.pprint(eqK2)
    print("")

    eqwc2 = eqwc.subs({A: 1E5, wp: 20 * smp.pi, K: K2}).evalf()
    wc2 = smp.solve(eqwc2, wc)[0]
    smp.pprint(eqwc2)
    print("")

    ex2interval = [0.5, 1, 5, 20, 50, 200, 500, 1000]
    ex2Vals = []

    for val in ex2interval:
    temp = Hjw_abs.subs({w: wc2 * val, wc: wc2, K: K2})
    ex2Vals.append(temp)
    print('para freq:', round(((val*wc2)/(2*smp.pi)).evalf(), 2),
    '\ttemos:', round(temp, 2))
    print("")


    # Plotando os graficos

    frequencias_plot = [i for i in range(1, 100000, 100)]

    plotH1 = [Hjw_abs.subs({w: i, wc: wc1, K: K1}) for i in frequencias_plot]
    plotH2 = [Hjw_abs.subs({w: i, wc: wc2, K: K2}) for i in frequencias_plot]

    fig, ax = plt.subplots()

    ax.plot(frequencias_plot, plotH1, color='blue', label='Exemplo 1')
    ax.plot(frequencias_plot, plotH2, color='orange', label='Exemplo 2')
    ax.legend(['Exemplo 1', 'Exemplo 2'])
    plt.xlabel('w rad/s')
    plt.ylabel('|H(jw)|')
    plt.title('Magnitude de H(jw)')


    plt.show()

    # Medicoes na pratica

    # Exemplo1

\end{python}
\documentclass[12pt,twoside, a4paper, twocolumn]{article}
\usepackage[utf8]{inputenc}
\usepackage[brazil]{babel}
\usepackage[margin = 0.5in]{geometry}
\usepackage{amsmath}
\usepackage{amsthm}
\usepackage{amssymb}
\usepackage{amsthm}
\usepackage{setspace}
\usepackage[americanvoltages,fulldiodes,siunitx]{circuitikz}
\usepackage{lipsum}
\usepackage{pgfplots}
\usepackage{ifthen}
\usepackage{adjustbox}
\usepackage[section]{placeins}
\usepackage{hyperref}

\pgfplotsset{compat=newest}



%  #1 color - optional #2 x_0 #3 y_0 #4 x_f #5 y_f #6 name - optional  #7 true if adding lines to axis

\newcommand{\drawvector} [9] [color=cyan] {
    \draw[line width=1.5pt,#1,-stealth](axis cs: #2, #3)--(axis cs: #4, #5) node[anchor=south west]{$#6$};

    

\ifthenelse{\equal{#7}{true}}{
    \draw[line width=1pt,#1, dashed](axis cs: #4, #5)--(axis cs: #4, 0) node[anchor= north west]{$#8$};
    \draw[line width=1pt,#1, dashed](axis cs: #4, #5)--(axis cs: 0, #5) node[anchor=south east]{$#9$};
    }
    {}
}

\newcommand\deriv[2]{\frac{\mathrm d #1}{\mathrm d #2}}


\title{Resumo Calculo 4 Primeira Unidade}
\author{Henrique da Silva \\ hpsilva@proton.me}
\date{\today}
\pgfplotsset{width = 10cm, compat = 1.9}


\begin{document}
\maketitle
\pagenumbering{gobble}
\newpage
%pagenumbering{roman}
\tableofcontents
\newpage

\section{Relacoes trigonometricas}

\subparagraph*{Todas relacoes trigonometricas podem ser tiradas destas 3 relacoes trigonometricas fundamentais:}
\begin{equation}
    \begin{aligned}
         & \sin(x)^2 + \cos(x)^2  = 1                                 \\
         & \sin(a+b)              = \sin(a) \cos(b) + \cos(a) \sin(b) \\
         & \cos(a+b)              = \cos(a) \cos(b) - \sin(a) \sin(b) \\
    \end{aligned}
\end{equation}

\subparagraph*{Vale a pena lembrar tambem que a funcao $\sin$ eh impar, e a funcao $\cos$ par. Ou seja:}

\begin{equation}
    \begin{aligned}
        \sin{-a} & = - \sin{a} \\
        \cos{-a} & = \cos{a}   \\
    \end{aligned}
\end{equation}

\section{Metodos de integracao}

\subsection{Fracoes parciais}
\subsection{Por partes}

\begin{equation}
    \int {u dv} = u v - \int{v du}
\end{equation}

\section{EDOs de primeira ordem}

\subparagraph*{Sao equacoes em que a incognita eh uma funcao de uma variavel, e a equacao involve a derivada da funcao}

\subparagraph*{Por exemplo:}

\begin{equation*}
    \begin{aligned}
        y'' - y' + y = x
    \end{aligned}
\end{equation*}

\subsection{Separavel}
\subparagraph*{Se conseguirmos separar a EDO de forma que Tudo que dependa de $y$ esteja de um lado e tudo que dependa de $x$ esteja do outro podemos resolver da seguinte forma:}

\begin{equation}
    \begin{aligned}
        \deriv{y}{x}         & = f(x)g(y)    \\
        \int \frac{dy}{g(y)} & = \int f(x)dx
    \end{aligned}
\end{equation}

\subsection{Fator integrante}

\subparagraph*{A ideia deste metodo eh achar um $\mu$ que multiplicando os dois lados da EDO, faca com que um lado vire uma regra do produto. }

\begin{equation}
    \begin{aligned}
        a(x)y' + b(x)y          & = c(x)               \\
        y' + \frac{b(x)}{a(x)}y & =  \frac{c(x)}{a(x)} \\
        y' + p(x)y              & = q(x)               \\
        \mu(y' + p(x)y          & = \mu q(x)           \\
        \mu y' + \mu p(x)y      & = \mu q(x)           \\
    \end{aligned}
\end{equation}

\subparagraph*{Para transformar o lado esquerdo em uma regra do produto no estilo $(fg)' = f' g + fg'$ precisamos que:}

\begin{equation}
    \begin{aligned}
        \deriv{\mu}{x} & = \mu p(x) \\
    \end{aligned}
\end{equation}

\subparagraph*{Que eh por si eh uma EDO separavel, etnao podemos resolve-la com o metodo das EDOs separaveis }

\begin{equation}
    \begin{aligned}
        \int{\frac{d\mu}{\mu}} & = \int{p(x)dx}      \\
        \ln{|\mu|}             & = \int{p(x)dx}  + C \\
    \end{aligned}
\end{equation}

\subparagraph*{Porem so preciso de uma unica solucao para isso, e nao todas, ja que estou apenas utilizando essa EDO para resolver uma outra EDO, logo posso escolher uma unica solucao:}

\begin{equation}
    \begin{aligned}
        \ln{\mu} & = \int{p(x)dx}                    \\
        \mu      & = \exp{\left(\int{p(x)dx}\right)}
    \end{aligned}
\end{equation}

\subparagraph*{E finalmente temos que:}

\begin{equation}
    \begin{aligned}
        \int{(\mu y)'} & = \int{\mu q} \\
        \mu y          & = \int{\mu q}
    \end{aligned}
\end{equation}

\subsection{Variavel homogenea}

\begin{equation}
    \begin{aligned}
        v  & = \frac{y}{x} \\
        y  & = xv          \\
        y' & = v' + xv'    \\
    \end{aligned}
\end{equation}

\subsection{Exatas}

\subparagraph*{Considerando:}

\begin{equation}
    M dx + N dy = 0
\end{equation}

\subparagraph*{Para a EDO ser exata, eh necessario que:}

\begin{equation}
    \deriv{M}{y} = \deriv{N}{x}
\end{equation}

\subparagraph*{Neste caso as solucoes sao a funcao potencial da EDO}

\subparagraph*{Por exemplo:}
\begin{equation}
    \begin{aligned}
         & (2x + y^2) + 2xy \deriv{y}{x} = 0 \\
         & (2x + y^2)\,dx + 2xy \,dy = 0     \\
         & f_x = (2x + y^2)                  \\
         & f_y = 2xy
    \end{aligned}
\end{equation}

\subparagraph*{Daqui eh so tirar uma funcao potencial e a igualar a uma constante}

\subsection{Fator integrante para exata}

\begin{equation}
    M dx + N dy = 0
\end{equation}

\subparagraph*{Se $M_y$ e $N_x$ nao forem exatas posso buscar um $\mu$ que a torne exata}

\begin{equation}
    \begin{aligned}
        (M \mu)_y         & = (N \mu)_x     \\
        \mu_y M + \mu M_y & = \mu_x N + N_x
    \end{aligned}
\end{equation}

\subparagraph*{Porem se assumirmos que $\mu$ so depende de $x$ ou de $y$ teremos respectivamenet:}

\section{EDOs de segunda ordem}

\subsection{Coefientes inteiros}

\subsubsection{Equacoes homogenea}

\subsubsection{Equacao nao homogenea}

\subparagraph*{Neste caso, eh achar duas solucoes para a equacao homogenea, e adicionar uma terceira solucao baseada em 1 bom chute.}



\end{document}


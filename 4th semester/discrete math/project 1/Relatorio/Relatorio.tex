\documentclass[12pt,twoside, a4paper, twocolumn]{article}
\usepackage[utf8]{inputenc}
\usepackage[brazil]{babel}
\usepackage[margin = 0.5in]{geometry}
\usepackage{amsmath}
\usepackage{amsthm}
\usepackage{amssymb}
\usepackage{amsthm}
\usepackage{setspace}
\usepackage[americanvoltages,fulldiodes,siunitx]{circuitikz}
\usepackage{lipsum}
\usepackage{pgfplots}
\usepackage{ifthen}
\usepackage{adjustbox}
\usepackage[section]{placeins}
\usepackage{hyperref}

\pgfplotsset{compat=newest}



%  #1 color - optional #2 x_0 #3 y_0 #4 x_f #5 y_f #6 name - optional  #7 true if adding lines to axis

\newcommand{\drawvector} [9] [color=cyan] {
    \draw[line width=1.5pt,#1,-stealth](axis cs: #2, #3)--(axis cs: #4, #5) node[anchor=south west]{$#6$};

    

\ifthenelse{\equal{#7}{true}}{
    \draw[line width=1pt,#1, dashed](axis cs: #4, #5)--(axis cs: #4, 0) node[anchor= north west]{$#8$};
    \draw[line width=1pt,#1, dashed](axis cs: #4, #5)--(axis cs: 0, #5) node[anchor=south east]{$#9$};
    }
    {}
}

\newcommand\deriv[2]{\frac{\mathrm d #1}{\mathrm d #2}}


\title{Primeiro Projeto de Matematica Discreta}
\author{Henrique da Silva \\ hpsilva@proton.me}
\date{\today}
\pgfplotsset{width = 10cm, compat = 1.9}


\begin{document}
\maketitle
\pagenumbering{gobble}
\newpage
%pagenumbering{roman}
\tableofcontents
\newpage



\section{Introdução}


\subparagraph*{Neste relatório, vamos discutir e implementar o sistema RSA.}

\subparagraph*{Para ter as ferramentas necessarias para construi-lo, primeiro precisamos construir algumas ferramentas, estas serao discutidas nas secoes 2, 3, 4, e 5.}

\subparagraph*{Todos arquivos utilizados para criar este relatorio, e o relatorio em si estão em:  \url{https://github.com/Shapis/ufpe_ee/tree/main/4th semester/}}

\section{O Codificador de texto}

\subparagraph*{Este foi criado para transformar uma $string$ de texto em um $int$, Atravez de dois metodos. TextoParaInteiro(string) e InteiroParaTexto(int). }

\subsection{TextoParaInteiro}

\subparagraph*{Este metodo recebe um texto e o torna em um m do tipo $int$ da seguinte maneira:}

\begin{equation}
    m = \sum_{i=0}^{N - 1} cod(a_i)*27^i
\end{equation}

\subparagraph*{Com \emph{a} ate \emph{z} sendo definidos como 1 ate 26, "espaco" sendo definido como 27.}

\subsection{InteiroParaTexto}

\subparagraph*{Para retornar o texto, este metodo recebe um inteiro $m$ e faz a seguinte operacao:}

\begin{equation}
    a_i    = cod\left(\frac{m}{27^i} \pmod{27}\right)
\end{equation}

\subparagraph*{Para todo $i$ que nao faca $m$ ser menor que 1}

\subsection{Restricoes e limitacoes}

\subparagraph*{A principal restricao eh que isto foi implementado usando o tipo $int$ do $C\#$ que tem 32 bits. Porem, ja que ele contem tanto numeros positivos quanto negativos o valor maximo dele eh de:}

\begin{equation}
    \frac{2^{32}}{2} - 1 = 2147483647
\end{equation}

\subparagraph*{Estamos codificando o texto de maneira que cada digito ocupa ate: $2^N = 27$, $N = \frac{\log{27}}{\log{2}}$ bits}

\subparagraph*{ Entao a quantidade maxima de bits ocupados eh simplesmente $N*L$}

\subparagraph*{Para o nosso caso em especifico, que o tipo $int$ tem $2^{31} -1$ de tamanho, ou seja, seguramente ate $31$ bits. Temos que:}

\begin{equation}
    L * \frac{\log{27}}{\log{2}} \le 31
\end{equation}

\subparagraph*{Que nos da $L = 6$, ou seja, podemos seguramente converter ate $6$ caracteres para tipo $int$ e converte-los de volta.
}

\subparagraph*{Vale notar, que isto eh um limite inferior de seguranca. Na verdade temos $6.3$ digitos disponiveis, que nos permitiria por exemplo, guardar e recuperar, uma frase de sete digitos do tipo $zzzzzd$, Mas para ter certeza. Tem de ser 6 ou menos digitos.}

\section{A classe \emph{BigNumber}}

\subparagraph*{Esta sera uma classe que armazenara os numeros que utilizarei para a criacao do RSA. }
\subparagraph*{Utilizarei como base para meu $BigNumber$ a classe $BigInteger$ do $C\#$, que tem limite de tamanho tao grande quanto couber na memoria do computador que o esta utilizando.}
\subparagraph*{Para o nossos fins, queremos um $BigNumber$ que tenha no maximo $2048$ bits. Entao para todas operacoes de $BigNumber$ incluindo a sua propria criacao, criarei um $SafetySizeCheck$ que caso o $BigNumber$ exceda $2048$ bits, ele ira lancar uma excecao e parar o programa com a mensagem de erro apropriada.}
\subparagraph*{Importante lembrar que inclui o zero no $BigNumber$, entao na verdade o limite superior dele fica da seguinte maneira:}

\begin{equation}
    BigNumber \le 2^{2048} - 1
\end{equation}

\subparagraph*{E tambem importante lembrar que todas operacoes de checagem de seguranca ocorreram \emph{apos} a operacao ser realizada.}

\subparagraph*{Ou seja, o programa permitira operacoes inseguras, desde que o $BigNumber$ resultante desta operacao insegura nao exceda $2048$ bits.}

\subsection{Multiplicacao de \emph{BigNumber}}

\subparagraph*{Aqui podemos observar o seguinte:}

\begin{equation}
    2^a * 2^b = 2^{a + b}
\end{equation}

\subparagraph*{Entao a multiplicacao de dois $BigNumber$ de tamanho $a$ e $b$, pode no maximo nos dar um $BigNumber$ de tamanho $a+b$}

\subsection{Soma de \emph{BigNumber}}

\subparagraph*{Neste caso temos o seguinte:}

\begin{equation}
    \begin{aligned}
        2^a + 2^a = 2*(2^a) = 2^1 * 2^a = 2^{a + 1}
    \end{aligned}
\end{equation}

\subparagraph*{Logo podemos concluir que no maximo a soma de dois numeros de tamanho $N$ bits dara um numero de tamanho $N+1$ bits.}

\section{Aritmetica Modular}

\subsection{AddMod}

\subparagraph*{As limitacoes aqui sao as mesmas da soma de dois $BigNumber$ como vimos acima em $(7)$. }
\subparagraph*{A funcao AddMod pode no maximo dar um $BigNumber$ de tamanho $N+1$ bits, $N$ sendo o tamanho do maior dos dois $BigNumber$.}

\subsection{MulMod}

\subparagraph*{Vimos acima em $(6)$ as limitacoes de multiplicacao de dois $BigNumber$.}

\subparagraph*{Entao no maximo a soma dos tamanhos em bits dos nossos $BigNumber$ deve dar $2048$ que eh o tamanho que escolhemos para o nosso $BigNumber$}

\subsection{ExpMod}

\subparagraph*{No caso da exponenciacao precisamos que o produto dos tamanhos dos dois $BigNumber$ seja menor que $2048$ }

\subsection{InvMod}

\subparagraph*{Para resolver a congruencia linear utilizamos uo algoritmo de euclides extendido. E a operacao de maxima ordem que utilizamos em todas operacoes eh a de multiplicacao de $BigNumber$ que descrevemos em $(6)$}

\subparagraph*{Logo, nossa limitacao para garantir que nao vamos exceder os 2048 bits do $BigNumber$ eh que a soma em pares, de $a$, $b$, e $n$ nao exceda $2048$ bits.}

\section{Busca por numeros primos}

\subparagraph*{utilizarei o metodo de Miller Rabin para testar a primalidade dos numeros.}

\subsection{Testando os numeros dados:}

\begin{center}
    \begin{tabular}{ |ccc| }
        \hline
        $2^{521}-1$ & $\rightarrow$ & primo     \\
        $2^{523}-1$ & $\rightarrow$ & nao primo \\
        $2^{607}-1$ & $\rightarrow$ & primo     \\
        \hline
    \end{tabular}
\end{center}

\subsection{Achando novos primos:}

\subparagraph*{Para achar novos numeros primos criarei um novo $BigNumber$ de tamanho $n$, e testarei numeros impares menores que este $BigNumber$ ate o teste de Miller Rabin me retornar que provavelmente eh um primo.}

\subparagraph*{Para adicionar um elemento de aleatoriedade. Apos checar um numero, vamos subtrair a este $2*x$ com $x$ variando entre $0$ e $n$ aleatoriamente.}

\subparagraph*{Esta maneira de gerar numeros aleatorios, me garante que todo numero gerado tera tamanho menor do que $n$ bits. Porem, ela me retorna numeros primos repetidos com alguma frequencia.}

\subparagraph*{Com alguns testes numeros, tive a chance de aproxidamente $1$ em $5000$ de obter dois numeros primos identicos. O que para um sistema que sofra ataques com certeza nao seria aleatoriedade suficiente. Mas para nossos propositos de testar a criacao de um sistema $RSA$ que nao vai sofrer ataques. Sera o suficiente.}

\section{O Sistema RSA}

\subparagraph*{Vamos utilizar de todas ferramentas que criamos acima para montar o nosso sistema.}

\subsection{Codificando numeros grandes}

\subparagraph*{Inicialmente, notemos que o \emph{Codificador de Texto} discutido na secao dois, tinha limitacao de utilizar o tipo $int$ de $31$ bits. Que nao sera suficiente para nossos propositos. }

\subparagraph*{Entao escrevi dois novos metodos \emph{TextoParaBigNumber}, e \emph{BigNumberParaTexto} }

\subparagraph*{Estes tendo as mesmas limitacoes porem alterando tamanho do nosso numero de $31$ bits para $2048$ bits.}

\subparagraph*{Resolvendo a equacao (4) para $2048$, teremos que o nosso $L$ deve se limitar a no maximo $430$ caracteres.}

\end{document}


\documentclass[12pt,twoside, a4paper, twocolumn]{article}
\usepackage[utf8]{inputenc}
\usepackage[brazil]{babel}
\usepackage[margin = 0.5in]{geometry}
\usepackage{amsmath}
\usepackage{amsthm}
\usepackage{amssymb}
\usepackage{amsthm}
\usepackage{setspace}
\usepackage[americanvoltages,fulldiodes,siunitx]{circuitikz}
\usepackage{lipsum}
\usepackage{pgfplots}
\usepackage{ifthen}
\usepackage{adjustbox}
\usepackage[section]{placeins}
\usepackage{hyperref}
 
\pgfplotsset{compat=newest}
 
 
 
%  #1 color - optional #2 x_0 #3 y_0 #4 x_f #5 y_f #6 name - optional  #7 true if adding lines to axis
 
\newcommand{\drawvector} [9] [color=cyan] {
   \draw[line width=1.5pt,#1,-stealth](axis cs: #2, #3)--(axis cs: #4, #5) node[anchor=south west]{$#6$};
 
  
 
\ifthenelse{\equal{#7}{true}}{
   \draw[line width=1pt,#1, dashed](axis cs: #4, #5)--(axis cs: #4, 0) node[anchor= north west]{$#8$};
   \draw[line width=1pt,#1, dashed](axis cs: #4, #5)--(axis cs: 0, #5) node[anchor=south east]{$#9$};
   }
   {}
}
 
\newcommand\deriv[2]{\frac{\mathrm d #1}{\mathrm d #2}}
 
 
\title{Primeiro Projeto de Matemática Discreta}
\author{Henrique da Silva \\ hpsilva@proton.me}
\date{\today}
\pgfplotsset{width = 10cm, compat = 1.9}
 
 
\begin{document}
\maketitle
\pagenumbering{gobble}
\newpage
%pagenumbering{roman}
\tableofcontents
\newpage



\section{Introdução}


\subparagraph*{Neste relatório, vamos discutir e implementar o sistema RSA.}

\subparagraph*{Para ter as ferramentas necessárias para construí-lo, primeiro precisamos construir algumas ferramentas, estas serao discutidas nas seções 2, 3, 4, e 5.}

\subparagraph*{A inspiração para escolha do projeto de chaves RSA é que algo que eu já utilizo diariamente, a autenticação feita pelo \emph{git} utiliza um par privado e público de chaves $RSA$. }

\subparagraph*{Objetivo do projeto é conseguir gerar um par seguro de chaves \emph{RSA} e utilizá-lo para fazer minha autenticação no \emph{git}.}

\subparagraph*{Todos arquivos utilizados para criar este relatório, e o relatorio em si estão em:  \url{https://github.com/Shapis/ufpe_ee/tree/main/4th semester/}}

\section{O Codificador de texto}

\subparagraph*{Este foi criado para transformar uma $string$ de texto em um $int$, através de dois métodos. TextoParaInteiro(string) e InteiroParaTexto(int). }

\subsection{TextoParaInteiro}

\subparagraph*{Este método recebe um texto e o torna em um m do tipo $int$ da seguinte maneira:}

\begin{equation}
    m = \sum_{i=0}^{N - 1} cod(a_i)*27^i
\end{equation}

\subparagraph*{Com \emph{a} ate \emph{z} sendo definidos como 1 ate 26, "espaco" sendo definido como 27.}

\subsection{InteiroParaTexto}

\subparagraph*{Para retornar o texto, este método recebe um inteiro $m$ e faz a seguinte operação:}

\begin{equation}
    a_i    = cod\left(\frac{m}{27^i} \pmod{27}\right)
\end{equation}

\subparagraph*{Para todo $i$ que nao faca $m$ ser menor que 1}

\subsection{Restrições e limitações}

\subparagraph*{A principal restrição é que isto foi implementado usando o tipo $int$ do $C\#$ que tem 32 bits. Porém, já que ele contém tanto números positivos quanto negativos o valor máximo dele é de:}

\begin{equation}
    \frac{2^{32}}{2} - 1 = 2147483647
\end{equation}

\subparagraph*{Estamos codificando o texto de maneira que cada dígito ocupa ate: $2^N = 27$, $N = \frac{\log{27}}{\log{2}}$ bits}

\subparagraph*{Então a quantidade máxima de bits ocupados eh simplesmente $N*L$}

\subparagraph*{Para o nosso caso em específico, que o tipo $int$ tem $2^{31} -1$ de tamanho, ou seja, seguramente ate $31$ bits. Temos que:}

\begin{equation}
    L * \frac{\log{27}}{\log{2}} \le 31
\end{equation}

\subparagraph*{Que nos dá $L = 6$, ou seja, podemos seguramente converter ate $6$ caracteres para tipo $int$ e convertê-los de volta.
}

\subparagraph*{Vale notar, que isto é um limite inferior de seguranca. Na verdade temos $6.3$ dígitos disponíveis, que nos permitiria por exemplo, guardar e recuperar, uma frase de sete dígitos do tipo $zzzzzd$, Mas para ter certeza. Tem de ser 6 ou menos dígitos.}

\section{A classe \emph{BigNumber}}

\subparagraph{Esta será uma classe que armazenará os números que utilizarei para a criação do RSA. }
\subparagraph*{Utilizarei como base para meu $BigNumber$ a classe $BigInteger$ do $C\#$, que tem limite de tamanho tão grande quanto couber na memória do computador que o está utilizando.}
\subparagraph*{Para o nossos fins, queremos um $BigNumber$ que tenha no maximo $2048$ bits. Então para todas operações de $BigNumber$ incluindo a sua própria criação, criarei um $SafetySizeCheck$ que caso o $BigNumber$ exceda $2048$ bits, ele irá lançar uma exceção e parar o programa com a mensagem de erro apropriada.}
\subparagraph*{Importante lembrar que inclui o zero no $BigNumber$, entao na verdade o limite superior dele fica da seguinte maneira:}

\begin{equation}
    BigNumber \le 2^{2048} - 1
\end{equation}

\subparagraph*{É também importante lembrar que todas operações de checagem de segurança ocorreram \emph{após} a operação ser realizada.}

\subparagraph*{Ou seja, o programa permitirá operações inseguras, desde que o $BigNumber$ resultante desta operação insegura não exceda $2048$ bits.}

\subsection{Multiplicacao de \emph{BigNumber}}

\subparagraph*{Aqui podemos observar o seguinte:}

\begin{equation}
    2^a * 2^b = 2^{a + b}
\end{equation}

\subparagraph*{Então a multiplicação de dois $BigNumber$ de tamanho $a$ e $b$, pode no máximo nos dar um $BigNumber$ de tamanho $a+b$}

\subsection{Soma de \emph{BigNumber}}

\subparagraph*{Neste caso temos o seguinte:}

\begin{equation}
    \begin{aligned}
        2^a + 2^a = 2*(2^a) = 2^1 * 2^a = 2^{a + 1}
    \end{aligned}
\end{equation}

\subparagraph*{Logo podemos concluir que no máximo a soma de dois números de tamanho $N$ bits dará um numero de tamanho $N+1$ bits.}

\subsection{Codificando numeros grandes}

\subparagraph*{Inicialmente, notamos que o \emph{Codificador de Texto} discutido na seção dois, tinha limitação de utilizar o tipo $int$ de $31$ bits. Que não será suficiente para nossos propósitos. }

\subparagraph*{Então escrevi dois novos métodos \emph{TextoParaBigNumber}, e \emph{BigNumberParaTexto} }

\subparagraph*{Estes tendo as mesmas limitações porém alterando o tamanho do nosso número de $31$ bits para $2048$ bits.}

\subparagraph*{Resolvendo a equação (4) para $2048$, teremos que o nosso $L$ deve se limitar a no máximo $430$ caracteres.}

\section{Aritmetica Modular}

\subsection{AddMod}

\subparagraph*{As limitações aqui sao as mesmas da soma de dois $BigNumber$ como vimos acima em $(7)$. }
\subparagraph*{A função AddMod pode no máximo dar um $BigNumber$ de tamanho $N+1$ bits, $N$ sendo o tamanho do maior dos dois $BigNumber$.}

\subsection{MulMod}

\subparagraph*{Vimos acima em $(6)$ as limitações de multiplicação de dois $BigNumber$.}

\subparagraph*{Então no máximo a soma dos tamanhos em bits dos nossos $BigNumber$ deve dar $2048$ que eh o tamanho que escolhemos para o nosso $BigNumber$}

\subsection{ExpMod}

\subparagraph*{No caso da exponencial precisamos que o produto dos tamanhos dos dois $BigNumber$ seja menor que $2048$ }

\subsection{InvMod}

\subparagraph*{Para resolver a congruência linear utilizamos o algoritmo de euclides estendido. E a operação de máxima ordem que utilizamos em todas operações eh a de multiplicação de $BigNumber$ que descrevemos em $(6)$}

\subparagraph*{Logo, nossa limitação para garantir que não vamos exceder os 2048 bits do $BigNumber$ eh que a soma em pares, de $a$, $b$, e $n$ não exceda $2048$ bits.}



\section{Busca por números primos}

\subparagraph*{utilizarei o método de Miller Rabin para testar a primalidade dos números.}

\subsection{Testando os números dados:}

\begin{center}
    \begin{tabular}{ |ccc| }
        \hline
        $2^{521}-1$ & $\rightarrow$ & primo     \\
        $2^{523}-1$ & $\rightarrow$ & nao primo \\
        $2^{607}-1$ & $\rightarrow$ & primo     \\
        \hline
    \end{tabular}
\end{center}

\subsection{Achando novos primos:}

\subparagraph*{Para achar novos números primos criarei um novo $BigNumber$ de tamanho $n$, e testarei numeros impares menores que este $BigNumber$ ate o teste de Miller Rabin me retorna que provavelmente é um primo.}

\subparagraph*{Para adicionar um elemento de aleatoriedade. Após checar um número, vamos subtrair a este $2*x$ com $x$ variando entre $0$ e $n$ aleatoriamente.}

\subparagraph*{Esta maneira de gerar números aleatórios, me garante que todo número gerado terá tamanho menor do que $n$ bits. Porém, ela me retorna números primos repetidos com alguma frequência.}

\subparagraph*{Com alguns testes de números, tive a chance de aproximadamente $1$ em $5000$ de obter dois números primos idênticos. O que para um sistema que sofre ataques com certeza não seria aleatoriedade suficiente. Mas para nossos propósitos de testar a criação de um sistema $RSA$ que não vai sofrer ataques. Será o suficiente.}



\section{O Sistema RSA}

\subparagraph*{Vamos utilizar de todas ferramentas que criamos acima para montar o nosso sistema.}



\subsection{Geração do par de chaves RSA}

\subparagraph*{Inicialmente vamos achar dois números primos, $p$, e $q$ de $512$ bits aleatórios.}

\subparagraph*{Vamos definir $n = pq$ e $\phi(n) = (p-1)(q-1)$}

\subparagraph*{Vamos também escolher um $\epsilon$ tal que $MDC(\epsilon, \phi(n)) = 1$. Para simplicidade, escolhi o menor $\epsilon$ para qual esse $MDC$ seja 1.}

\subparagraph*{E através do sistema de resolução de congruências lineares que criamos na seção 4.4, vamos computar um $d$ da seguinte maneira:}

\begin{equation}
    \epsilon d \equiv 1 (mod \phi(n))
\end{equation}

\subparagraph*{Então, afinal temos um $\epsilon$ e um $n$ que serão nossa chave pública, e um $d$ que será nossa chave privada.}

\subparagraph*{Ambos chaves privadas e públicas serão salvas em arquivo.}

\subsection{Codificando e decodificando}

\subparagraph*{Vamos utilizar do nosso \emph{Codificador de Texto} nesta etapa.}

\subparagraph*{Com tudo que temos em mãos, os passos para codificação e decodificação são simples como veremos a seguir:}

\subsubsection{Codificacao}

\begin{equation}
    Y \equiv X^{\epsilon} \pmod{n}
\end{equation}

\subparagraph*{Onde X é a mensagem em texto limpo. Y a mensagem criptografada, e $\epsilon$ e $n$ são a chave pública.}

\subparagraph*{Isto foi implementado na classe $RSA$, e para simplificar, $RSA.Codificar$ recebe textos limpos e a chave pública, e retorna textos criptografados ao invés de cifrar o $BigNumber$}

\subsubsection{Decodificando}

\begin{equation}
    Z \equiv Y^{d} \pmod{n}
\end{equation}

\subparagraph*{Onde Z é a mensagem em decodificada. Y a mensagem criptografada, e $d$ é a chave privada.}

\subparagraph*{Importante de notar que o seguinte eh verdade:}

\begin{equation}
    Z \equiv X \pmod{n}
\end{equation}

\subparagraph*{Isto foi implementado na classe $RSA$, e para simplificar, $RSA.Decodificar$ recebe textos cifrados e a chave privada, e retorna textos limpos}

\subsection{Testando para um texto pré-determinado}

\subparagraph*{Quando executado o programa irá mostrar um texto ao usuário, codificá-lo, mostrar o texto codificado, e após isso decodificar e mostrar o texto decodificado. O texto decodificado deve ser o mesmo que o texto original.}

\subparagraph*{O embasamento teorico disto esta descrito na secao 6.2.}

\subsection{Testado para texto arbitrario Y}

\subparagraph*{O programa irá pedir ao usuário um texto, vamos codificar este texto, exibi-lo codificado, e após isso decodificar e exibi-lo novamente. O texto decodificado deve ser o mesmo que o texto original.}

\subparagraph*{Lembrando que há limitação de tamanho do texto em $430$ caracteres como visto na seção 3.3}

\subsection{Assinatura digital}

\subparagraph*{Para assinar um documento $X$, basta divulgar o $Y$ de:}

\begin{equation}
    Y \equiv X^{d} \pmod{n}
\end{equation}

\subparagraph*{E para verificar, fazemos o oposto que é:}

\begin{equation}
    Z \equiv Y^{ d} \pmod{n}
\end{equation}

\subparagraph*{Lembrando da equação (10) que nos dá que $Z \equiv X$ já que:}

\begin{equation}
    Z \equiv X^{de} \equiv X \pmod{n}
    asdas
\end{equation}

\subparagraph*{Nesta etapa, o programa irá assinar o texto que o usuário escreveu anteriormente, e verificar a assinatura.}

\subparagraph*{Importante notar que assinatura é um caso parecido com o de codificação de texto. Exceto que neste caso vamos cifrar o texto limpo com nossa chave privada, e decodificar o texto cifrado com a chave pública.}

\subsection{Autenticação de remetente e destinatário}

\subparagraph*{Neste caso, temos dois usuários $A$ e $B$. }

\subparagraph*{O usuário $A$ fará a codificação da mensagem com a chave pública de $B$, assim garantido que só $B$ lerá a mensagem, e envia junto com a mensagem a sua assinatura digital.}

\subparagraph*{O usuário $B$ usara sua chave privada para decodificar a mensagem, e usara a chave pública de $A$ para verificar a assinatura digital.}

\subparagraph*{O programa vai guiar o usuário A em enviar uma mensagem assinada para o usuário B, e o usuário B verificará que a mensagem foi assinada por A.}

\section{Conclusões}

\subparagraph*{Neste projeto conseguimos criar um sistema simples que criptografa, descriptografa mensagens de texto, e assina mensagens.}

\subparagraph*{As maiores vulnerabilidades que enxergamos no projeto sao a geração dos números primos, ja que o método "aleatório" que usamos não foi um método extremamente robusto, havendo a possibilidade de repetição de números primos com probabilidade em torno de $1/5000$}

\subparagraph*{Também há problemas com a segurança do arquivo de texto que contém as chaves privadas. Qualquer acesso indevido a ele, permitiria leitura das mensagens por terceiros.}

\subparagraph*{Mas em geral, creio que atingimos os objetivos que foram descritos na introdução do projeto.}

\subparagraph*{Alterei as chaves que uso para acesso a git por ssh para chaves geradas por este programa, mesmo que haja estes problemas descritos acima. }

\subparagraph*{Creio que o maior ganho de segurança será trabalhando na
    aleatoriedade dos primos gerados.}


\end{document}
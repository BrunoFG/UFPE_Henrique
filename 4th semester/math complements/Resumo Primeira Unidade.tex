\documentclass[12pt,twoside, a4paper, twocolumn]{article}
\usepackage[utf8]{inputenc}
\usepackage[brazil]{babel}
\usepackage[margin = 0.5in]{geometry}
\usepackage{amsmath}
\usepackage{amssymb}
\usepackage{amsthm}
\usepackage{setspace}
\usepackage{circuitikz}
\usepackage{lipsum}
\usepackage{pgfplots}



\title{Resumo Complementos de Matematica Primeira Unidade}
\author{Henrique da Silva \\ hpsilva@proton.me}
\date{\today}
\pgfplotsset{width = 10cm, compat = 1.9}


\begin{document}
\maketitle
\pagenumbering{gobble}
\newpage
%pagenumbering{roman}
\tableofcontents
\newpage

\newcommand\deriv[2]{\frac{\mathrm d #1}{\mathrm d #2}}

\section{Primeira Eq. de Maxwell}
\paragraph{Essa equacao vem da lei de Gauss e diz que o fluxo eletrico eh dado pela seguinte equacao:
}
\begin{equation}
    \oint \vec{E} \, * \, \vec{dA} = \frac{q}{\epsilon_0}
\end{equation}


\end{document}
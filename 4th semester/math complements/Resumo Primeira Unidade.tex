\documentclass[12pt,twoside, a4paper, twocolumn]{article}
\usepackage[utf8]{inputenc}
\usepackage[brazil]{babel}
\usepackage[margin = 0.5in]{geometry}
\usepackage{amsmath}
\usepackage{amsthm}
\usepackage{amssymb}
\usepackage{amsthm}
\usepackage{setspace}
\usepackage{circuitikz}
\usepackage{lipsum}
\usepackage{pgfplots}



\title{Resumo Complementos de Matematica Primeira Unidade}
\author{Henrique da Silva \\ hpsilva@proton.me}
\date{\today}
\pgfplotsset{width = 10cm, compat = 1.9}


\begin{document}
\maketitle
\pagenumbering{gobble}
\newpage
%pagenumbering{roman}
\tableofcontents
\newpage

\newcommand\deriv[2]{\frac{\mathrm d #1}{\mathrm d #2}}

\section{Potencias de i}
\paragraph{As potencias de i sao periodicas em 4. Da seguinte maneira:
}
\begin{center}
    \begin{tabular}{ |ccc| }
        \hline
        $i^0$ & $=$      & 1            \\
        $i^1$ & $=$      & $i$          \\
        $i^2$ & $=$      & -1           \\
        $i^3$ & $=$      & $-i$         \\
        $i^4$ & $=$      & 1            \\
        $i^5$ & $=$      & $i$          \\
        $i^6$ & $=$      & $-1$         \\
              & $\vdots$ &              \\
        $i^n$ & $=$      & $i^{n \% 4}$ \\
        \hline
    \end{tabular}
\end{center}
\paragraph{Com "\%" sendo resto da divisao inteira
}
% \begin{equation}
%     \oint \vec{E} \, * \, \vec{dA} = \frac{q}{\epsilon_0}
% \end{equation}

\section{Forma algebrica de um numero complexo}
\paragraph{A forma algebrica de um numero complexo eh:}
\begin{equation}
    z = a + ib
\end{equation}
% \paragraph{$a + ib \in \mathbb{C}$
% }


\end{document}
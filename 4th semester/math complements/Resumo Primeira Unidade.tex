\documentclass[12pt,twoside, a4paper, twocolumn]{article}
\usepackage[utf8]{inputenc}
\usepackage[brazil]{babel}
\usepackage[margin = 0.5in]{geometry}
\usepackage{amsmath}
\usepackage{amsthm}
\usepackage{amssymb}
\usepackage{amsthm}
\usepackage{setspace}
\usepackage{circuitikz}
\usepackage{lipsum}
\usepackage{pgfplots}
\usepackage{ifthen}
\usepackage{adjustbox}

\pgfplotsset{compat=newest}

%  #1 color - optional #2 x_0 #3 y_0 #4 x_f #5 y_f #6 name - optional  #7 true if adding lines to axis

\newcommand{\drawvector} [9] [color=cyan] {
    \draw[line width=1.5pt,#1,-stealth](axis cs: #2, #3)--(axis cs: #4, #5) node[anchor=south west]{$#6$};

    

\ifthenelse{\equal{#7}{true}}{
    \draw[line width=1pt,#1, dashed](axis cs: #4, #5)--(axis cs: #4, 0) node[anchor= north west]{$#8$};
    \draw[line width=1pt,#1, dashed](axis cs: #4, #5)--(axis cs: 0, #5) node[anchor=south east]{$#9$};
    }
    {}
}




\title{Resumo Complementos de Matematica Primeira Unidade}
\author{Henrique da Silva \\ hpsilva@proton.me}
\date{\today}
\pgfplotsset{width = 10cm, compat = 1.9}


\begin{document}
\maketitle
\pagenumbering{gobble}
\newpage
%pagenumbering{roman}
\tableofcontents
\newpage

\newcommand\deriv[2]{\frac{\mathrm d #1}{\mathrm d #2}}

\section{Potencias de i}
\paragraph{As potencias de i sao periodicas em 4. Da seguinte maneira:
}
\begin{center}
    \begin{tabular}{ |ccc| }
        \hline
        $i^0$ & $=$      & 1            \\
        $i^1$ & $=$      & $i$          \\
        $i^2$ & $=$      & -1           \\
        $i^3$ & $=$      & $-i$         \\
        $i^4$ & $=$      & 1            \\
        $i^5$ & $=$      & $i$          \\
        $i^6$ & $=$      & $-1$         \\
              & $\vdots$ &              \\
        $i^n$ & $=$      & $i^{n \% 4}$ \\
        \hline
    \end{tabular}
\end{center}
\paragraph{Com "\%" sendo resto da divisao inteira
}
% \begin{equation}
%     \oint \vec{E} \, * \, \vec{dA} = \frac{q}{\epsilon_0}
% \end{equation}

\section{Forma algebrica de um numero complexo}
\paragraph{A forma algebrica de um numero complexo eh:}
\begin{equation}
    Z = a + ib
\end{equation}
\paragraph{Onde $a$ eh a componente real de $Z$ e pode ser chamado de $Re(Z)$ e $b$ eh a componente imaginaria e pode ser chamado de $Im(Z)$ }

\paragraph*{Podemos dizer que os numeros $\Re$ sao um subconjunto de $\mathbb{C}$, exceto que no caso de um numero $\Re$ a parte imaginaria $b$ seria 0, alguns exemplos:}
\begin{center}
    \begin{tabular}{ |c|c|c| }
        \hline
        $5 + i$    & $a = 5$  & $b = 1$  \\
        $4 - 3i^2$ & $a = 4$  & $b = -3$ \\
        $12$       & $a = 12$ & $b = 0$  \\
        $7i^3$     & $a = 0$  & $b = 7$  \\
        \hline
    \end{tabular}
\end{center}
\paragraph{Dois numeros complexos sao iguais se seus componentes reais e imaginarios forem iguais}

\section{Operacoes na forma algebrica}

\paragraph*{Nos exemplos a seguir: $Z_n = a_n + ib_n$}

\subsection{Adicao $\&$ Subtracao}
\subparagraph{Para subtrair e adicionar basta subtrair e adicionar as partes imaginarias dos numeros complexos}

\begin{equation}
    Z_1 + Z_2 = (a_1+a_2) + (b_1+b_2)i
\end{equation}
\begin{equation}
    Z_1 - Z_2 = (a_1-a_2) + (b_1-b_2)i
\end{equation}

\subsection{Multiplicacao}
\subparagraph{Vamos utilizar a distributividade e o fato que $i^2 = -1$}
\begin{equation*}
    \begin{aligned}
        Z_1 * Z_2 & = (a_1+b_1i)(a_2+b_2i)                    \\
        Z_1 * Z_2 & = a_1a_2 + a_1 b_2i + b_1a_2i + b_1b_2i^2 \\
    \end{aligned}
\end{equation*}
\subparagraph*{Como $i^2 = -1$ podemos entao simplificar em:}
\begin{equation}
    Z_1 * Z_2  = (a_1a_2- b_1b_2) + (a_1 b_2 + b_1a_2)i
\end{equation}

\subsection{Divisao}
\subparagraph{O conjugado de $Z$ eh $\overline{Z}$. Se $Z = a + ib$ entao  $\overline{Z} = a - ib$}
\subparagraph{Algo interessante acontece quando fazemos $Z * \overline{Z}$:}
\begin{equation}
    \begin{aligned}
        Z_1 * \overline{Z} & = (a+bi)(a-bi) \\
        Z_1 * \overline{Z} & = a^2 - (bi)^2 \\
        Z_1 * \overline{Z} & = a^2 - b^2i^2 \\
        Z_1 * \overline{Z} & = a^2 + b^2
    \end{aligned}
\end{equation}
\subparagraph{Ou seja, essa operacao nos da um escalar. E vamos utilizar disso para poder fazer a divisao.}
\subparagraph{Para fazer a divisao de $Z$ por $\overline{Z}$ fazemos:}
\begin{equation}
    \begin{aligned}
        \frac{Z}{W} = \frac{Z}{W} * \frac{\overline{W}}{\overline{W}}
    \end{aligned}
\end{equation}
\subparagraph{E transformamos a operacao de divisao de numeros complexos em uma multiplicacao de $Z$ por $\overline{W}$ divido por um escalar $W*\overline{W}$}


\section{Representacao geometrica}

\paragraph{Na representacao geometrica de um numero complexo podemos ver o eixo $x$ como a parte real e o $y$ como a parte imaginaria, como no exemplo abaixo:}
\paragraph*{}


\begin{adjustbox}{scale=0.9}
    \begin{tikzpicture}
        \begin{axis}[
            clip = false,
            xmin=0, xmax=2.5,
            ymin=-1.5, ymax=1.5,
            axis lines=center,
            xlabel = $x \, REAL$, ylabel=$y \, IMAGINARIO$,
            title={ Z = a + bi},
            xtick={},
            xticklabels={}
            ytick={},
            yticklabels={}
            % xticklabel style = {anchor=south west},
            % xmajorgrids=true,
            % ymajorgrids=true,
            % grid style=dashed,
            ]


            % \addplot[domain=0:2*pi,color=blue, samples=100]{sin(deg(x))}
            % node[anchor=west, pos =0.7] {$3120$};



            \drawvector{0}{0}{1}{1}{Z}{true}{a}{b}
            \drawvector{0}{0}{1}{-1}{\overline{Z}}{true}{a}{-b}


        \end{axis}
    \end{tikzpicture}
\end{adjustbox}

\paragraph*{Eh interessante notar que $\overline{Z}$ eh o simetrico de $Z$, porem oposto no eixo imaginario}

\section{Modulo}

\paragraph*{Como temos uma representacao geometrica do numero complexo, podemos calcular o modulo do numero complexo por simplesmente a hipotenusa de $a$ e $b$ do mesmo jeito que fariamos com um vetor comum}
\paragraph*{O modulo sera representado pela letra $\rho$ ou por $| Z |$}

\begin{equation}
    \begin{aligned}
        \rho = | Z | = \sqrt{a^2 + b^2} = | \overline{Z} |
    \end{aligned}
\end{equation}
\pagebreak
\section{Argumento}

\paragraph*{O argumento eh simplesmente o angulo entre o eixo real e o vetor complexo sera representado pela letra $\theta$ ou por $arg(Z)$ }
\paragraph*{}

\begin{adjustbox}{scale=0.9}
    \begin{tikzpicture}
        \begin{axis}[
            clip = false,
            xmin=0, xmax=2,
            ymin=0, ymax=2,
            axis lines=center,
            xlabel = $x \, REAL$, ylabel=$y \, IMAGINARIO$,
            title={ Z = a + bi},
            xtick={},
            xticklabels={}
            ytick={},
            yticklabels={}
            % xticklabel style = {anchor=south west},
            % xmajorgrids=true,
            % ymajorgrids=true,
            % grid style=dashed,
            ]


            % \addplot[domain=0:2*pi,color=blue, samples=100]{sin(deg(x))}
            % node[anchor=west, pos =0.7] {$3120$};



            \drawvector{0}{0}{1}{1}{Z}{true}{a}{b};
            \draw[line width=0pt,cyan,-stealth](axis cs: 0, 0)--(axis cs: 1, 1)  node[anchor= south east, pos =0.5] {$\rho$};


            \draw [cyan](axis cs: 0.3, 0) arc[start angle=0, end angle=45, radius=30]
            node[anchor= west, pos =0.6] {$\theta$};


        \end{axis}
    \end{tikzpicture}
\end{adjustbox}

\paragraph*{Dai temos:}

\begin{equation}
    \begin{aligned}
        \cos{\theta} & = \frac{a}{\rho} \\
        \sin{\theta} & = \frac{b}{\rho}
    \end{aligned}
\end{equation}

\section{Complexo na forma polar}

\paragraph*{Entendendo o conceito de argumento, podemos entao naturalmente:}

\begin{equation}
    \begin{aligned}
        \cos{\theta} & = \frac{a}{\rho} \rightarrow a =\rho cos{\theta} \\
        \sin{\theta} & = \frac{b}{\rho} \rightarrow b =\rho sin{\theta}
    \end{aligned}
\end{equation}

\paragraph*{Que nos da $a$ e $b$ no formato de coordenadas polares, entao podemos re-escrever o nosso $Z$ como:}

\begin{equation}
    \begin{aligned}
        Z & = \rho * cos{\theta} + \rho * sin{\theta} * i        \\
        Z & = \rho * \left(cos{\theta} +  sin{\theta} * i\right)
    \end{aligned}
\end{equation}

\section{Operacoes na forma polar}

\paragraph*{As simplificacoes abaixo vem diretamente da multiplicacao e divisao de numeros complexos na forma polar, e simplificacao por regras de trigonometria.}
\subsection{Multiplicacao}
\begin{equation}
    \begin{aligned}
        Z_1 * Z_2 = \rho_1 \rho_2  \left[\cos(\theta_1+\theta_2) + i  \sin(\theta_1+\theta_2)\right] \\
    \end{aligned}
\end{equation}

\subsection{Divisao}
\begin{equation}
    \begin{aligned}
        \frac{Z_1}{Z_2} = \frac{\rho_1}{\rho_2}   \left[\cos(\theta_1-\theta_2) + i  \sin(\theta_1-\theta_2)\right] \\
    \end{aligned}
\end{equation}

\section{Forma de Moivre}
\paragraph{Com tudo que foi visto acima vamos agora fazer a seguinte definicao:}
\subsection{Definicao}
\begin{equation}
    \begin{aligned}
        e^{it} = \cos{t} + i \sin{t}
    \end{aligned}
\end{equation}
\paragraph{Escrever nessa forma nos da varias propriedades uteis:}
\subsection{Propriedades}
\begin{equation*}
    \begin{aligned}
         & Z             = \rho * e^{it}                                         \\
         & \overline{Z}  = \rho * e^{i * \left(-t\right)}                        \\
         & | e^{it} |    = 1      \rightarrow \sqrt[]{\cos^2{t} + \sin^2{t}} = 1 \\
         & \frac{1}{e^{it}  } = e^{-it} = \overline{e^{it}  }                    \\
         & e^{it}  = e^{i*(t+2K\pi)}                                             \\
         & e^{it}  * e^{ig} = e^{i* (t+g)}
    \end{aligned}
\end{equation*}

\paragraph*{E por fim chegamos a uma forma mais simples de efetuar multiplicacoes e divisoes:}

\subsection{Multiplicacao}
\begin{equation}
    Z_1 * Z_2  = \rho_1  \rho_2 \, e^{i(t_1 + t_2)}
\end{equation}
\subsection{Divisao}
\begin{equation}
    \frac{Z_1}{Z_2}  = \frac{\rho_1  }{\rho_2}  \, e^{i(t_1 - t_2)}
\end{equation}

\subsection{Radiciacao}
\subparagraph*{Dado um numero complexo $Z$ qualquer,  $\sqrt[n]{Z}$ tera $n$ raizes complexas }

\subparagraph*{Podemos dizer a partir das propriedades o seguinte:}

\begin{equation*}
    \begin{aligned}
        Z                          & = \rho * e^{i (\theta + 2 K \pi)}        \\
        \sqrt[n]{Z}                & = W                                      \\
        Z                          & = W^n                                    \\
        W^n                        & =  \left(\rho_k * e^{i\theta_k}\right)^n \\
        W^n                        & =\rho_k ^n e^{i \theta_k n}              \\
        \rho_k ^n e^{i \theta_k n} & = \rho * e^{i (\theta + 2 K \pi)}        \\
    \end{aligned}
\end{equation*}

\subparagraph*{Entao conseguimos algumas conclusoes interessantes, primeiro que o nosso $\rho_k ^n$ nao depende de $k$, ou seja. para todas raizes teremos o mesmo modulo, so o que vai se alterar eh o argumento, ou seja, o angulo}

\begin{equation}
    \begin{aligned}
        \rho^n_k & = \rho           \\
        \rho_k   & = \sqrt[n]{\rho} \\
    \end{aligned}
\end{equation}

\subparagraph*{E segundo, concluimos que o nosso argumento varia de acordo com K da seguinte forma:}

\begin{equation}
    \begin{aligned}
        \theta_k n & = \theta + 2 K \pi           \\
        \theta_k   & = \frac{\theta + 2 K \pi}{n} \\
    \end{aligned}
\end{equation}

\subparagraph*{Concluindo, para conseguir cada raiz, basta fazermos o $K$ variar de $0$ a $n-1$, e eh util notar que isso vai criar vetores de tamanho $\rho$ equidistantes no intervalo $[0,2\pi]$}


% \begin{tikzpicture}

%     \begin{axis}[
%             clip = false,
%             xmin=0, xmax=pi*2.5,
%             ymin=0, ymax=2,
%             axis lines=left,
%             xlabel = $Voltage$, ylabel=$Time$,
%             title={abc},
%             xtick={0,pi/2,pi,3*pi/2, 2*pi},
%             xticklabels={$0$,$\frac{\pi}{2}$,$\pi$,$\frac{3\pi}{2}$,$2\pi$},
%             % xticklabel style = {anchor=south west},
%             % xmajorgrids=true,
%             % ymajorgrids=true,
%             % grid style=dashed,
%         ]

%  

%         \addplot[domain=0:2*pi,color=blue, samples=100]{sin(deg(x))}
%         node[anchor=west, pos =0.7] {$3120$};

%         \drawvector{0}{0}{1}{1}{}


%     \end{axis}

% \end{tikzpicture}

% \begin{tikzpicture}
%     \draw[thin,gray!40] (-2,-2) grid (2,2);
%     \draw[<->] (-2,0)--(2,0) node[right]{$x$};
%     \draw[<->] (0,-2)--(0,2) node[above]{$y$};
%     \draw[line width=2pt,blue,-stealth](0,0)--(1,2) node[anchor=south west]{$\boldsymbol{u}$};
%     \draw[line width=2pt,red,-stealth](0,0)--(-1,-1) node[anchor=north east]{$\boldsymbol{-u}$};
% \end{tikzpicture}

\section{Inversao complexa}

\section{asds}

\end{document}
\documentclass[12pt,twoside, a4paper, twocolumn]{article}
\usepackage[utf8]{inputenc}
\usepackage[brazil]{babel}
\usepackage[margin = 0.5in]{geometry}
\usepackage{amsmath}
\usepackage{amsthm}
\usepackage{amssymb}
\usepackage{amsthm}
\usepackage{setspace}
\usepackage[americanvoltages,fulldiodes,siunitx]{circuitikz}
\usepackage{lipsum}
\usepackage{pgfplots}
\usepackage{ifthen}
\usepackage{adjustbox}
\usepackage[section]{placeins}

\pgfplotsset{compat=newest}



%  #1 color - optional #2 x_0 #3 y_0 #4 x_f #5 y_f #6 name - optional  #7 true if adding lines to axis

\newcommand{\drawvector} [9] [color=cyan] {
    \draw[line width=1.5pt,#1,-stealth](axis cs: #2, #3)--(axis cs: #4, #5) node[anchor=south west]{$#6$};

    

\ifthenelse{\equal{#7}{true}}{
    \draw[line width=1pt,#1, dashed](axis cs: #4, #5)--(axis cs: #4, 0) node[anchor= north west]{$#8$};
    \draw[line width=1pt,#1, dashed](axis cs: #4, #5)--(axis cs: 0, #5) node[anchor=south east]{$#9$};
    }
    {}
}

\newcommand\deriv[2]{\frac{\mathrm d #1}{\mathrm d #2}}


\title{Resumo Circuitos 1 Primeira Unidade}
\author{Henrique da Silva \\ hpsilva@proton.me}
\date{\today}
\pgfplotsset{width = 10cm, compat = 1.9}


\begin{document}
\maketitle
\pagenumbering{gobble}
\newpage
%pagenumbering{roman}
\tableofcontents
\newpage

\section{Voltagem, corrente e resistencia}

\subsection{Corrente $I$}
\subparagraph*{Eh o fluxo de eletrons em um circuito, vamos por definicao dizer que a corrente se move do positivo para o negativo, ou seja, ao contrario da direcao do fluxo de eletrons.}
\subparagraph*{Sua unidade eh o \emph{Ampere} ou $A$ que eh simplemente Couloumbs por segundo $A = \frac{C}{s}$}
\subparagraph*{}

\subsection{Voltagem}
\subparagraph*{Voltagem $V$ eh o potencial que faz com que a corrente flua}


\subsection{Resistencia}
\subparagraph*{Eh algo que opoem o fluxo de corrente em um circuito e sua unidade eh o Ohm $\varOmega$}

\section{Circuitos}
\paragraph{Eh um caminho pelo qual passa uma corrente, pode ser aberto ou fechado}


\subsection{Circuito aberto}
\subparagraph*{}
\begin{center}
    \begin{circuitikz}
        \draw
        (0,0) to[battery1,  invert] (0,4) % l=5<\milli\volt>
        -- (4,4) -- (4,2.5)
        (4, 1.5) -- (4,0) -- (0,0)
        ;
        \draw (0,-0.05)
        node[rground]{};
    \end{circuitikz}
\end{center}

\subsection{Circuito fechado}
\subparagraph*{}
\begin{center}
    \begin{circuitikz}
        \draw
        (0,0) to[battery1,  invert] (0,4) % l=5<\milli\volt>
        -- (4,4)
        -- (4,0) -- (0,0)
        ;
        \draw (0,-0.05)
        node[rground]{};
    \end{circuitikz}
\end{center}

\subsection{Curto circuito}
\subparagraph*{Occorre quando ha um caminho de baixa resistencia impedindo da corrente passar pelo caminho que voce deseja que ela passe, no exemplo abaixo a corrente passara pelo caminho central ao inves de pelo resistor}
\subparagraph*{}
\begin{center}
    \begin{circuitikz}
        \draw
        (0,0) to[battery1,  invert] (0,4) % l=5<\milli\volt>
        -- (4,4)
        to[resistor] (4,0) -- (0,0)
        (2,4) -- (2,0)
        ;
        \draw (0,-0.05)
        node[rground]{};
    \end{circuitikz}
\end{center}

\section{Componentes de um circuito}
\paragraph*{Aqui estao as descricoes de alguns tipos de componentes que podem ser utilizados em um circuito}

\subsection{Fonte de voltagem}
\subparagraph*{Eh um componente que aumenta a voltagem de um lado para o outro dele, no exemplo abaixo, voce pode ver a fonte como uma bateria que aumenta em 5 $V$ o potencial do lado negativo para o positivo}
\begin{center}
    \begin{circuitikz}
        \draw
        (0,0) to[battery1,  invert,  l=5<\volt>] (0,3) % l=5<\milli\volt>

        (2,0) to[battery,  invert,  l=5<\volt>] (2,3) % l=5<\milli\volt>
        ;
    \end{circuitikz}
\end{center}


\subsection{Resistor}
\subparagraph*{Eh um elemento resistivo, basicamente todos elementos que precisam realizar trabalho com a corrente sao resistores, por exemplo tvs, ventiladores, lampadas etc. Sua unidade eh o \emph{Ohm} ou $\varOmega$}
\begin{center}
    \begin{circuitikz}
        \draw
        (0,0) to[resistor,   l=5<\ohm>] (0,3.5) % l=5<\milli\volt>

        ;
    \end{circuitikz}
\end{center}

\subsection{Capacitor}
\subparagraph*{Eh um elemento que guarda corrente na forma de um \emph{campo eletrico} e sua unidade eh o \emph{Farad} ou $F$}
\subparagraph*{}
\begin{center}
    \begin{circuitikz}
        \draw
        (0,0) to[capacitor,    l=5<\micro\farad>] (0,3.5) % l=5<\milli\volt>
        ;
    \end{circuitikz}
\end{center}

\subsection{Indutor}
\subparagraph*{Eh um elemento que guarda corrente na forma de um \emph{ campo magnetico} e sua unidade eh o \emph{Henry} ou $H$}
\subparagraph*{}
\begin{center}
    \begin{circuitikz}
        \draw
        (0,0) to[inductor,    l=5<\henry>] (0,3.5) % l=5<\milli\volt>
        ;
    \end{circuitikz}
\end{center}

\pagebreak
\subsection{Diodo}
\subparagraph*{Este elemento so perte que a corrente passe em uma direcao }
\subparagraph*{}
\begin{center}
    \begin{circuitikz}
        \draw
        (0,0) to[diode] (0,3.5) % l=5<\milli\volt>
        ;
    \end{circuitikz}
\end{center}

\subsection{LED - Light Emitting Diode}
\subparagraph*{Eh um elemento que quando a corrente passa por ele, ele emite luz}
\subparagraph*{}
\begin{center}
    \begin{circuitikz}
        \draw
        (0,0) to[led] (0,3.5) % l=5<\milli\volt>
        ;
    \end{circuitikz}
\end{center}

\subsection{Transistor}
\subparagraph*{Ele eh utilizado como um switch ou como um amplificador de sinal}
\subparagraph*{}
\begin{center}
    \begin{circuitikz}
        \draw (0,0) node[npn, rotate=180](Q){};
    \end{circuitikz}
\end{center}

\subsection{Transformador}
\subparagraph*{Permite que uma corrente seja convertida em outra corrente dentro de um circuito}
\subparagraph*{}
\begin{center}
    \begin{circuitikz}
        (0,0) to[transformer] (0,3.5) % l=5<\milli\volt>
    \end{circuitikz}
\end{center}

\section{Lei de Ohm}


\end{document}

